\chapter{FUNDAMENTAÇÃO TEÓRICA}
\label{chap:fundamentacao-teorica}

Este capítulo apresenta os fundamentos teóricos necessários para compreender a aplicação de modelos de machine learning na predição de vibração em moinhos de rolos verticais. A fundamentação está estruturada de forma a contextualizar os equipamentos estudados, os fenômenos físicos envolvidos, as técnicas de aprendizado de máquina aplicáveis e as métricas utilizadas para comparação de desempenho dos modelos.

\section{Análise de Vibração em Equipamentos Rotativos}
\label{sec:analise-vibracao}

A análise de vibração constitui uma das técnicas fundamentais da manutenção preditiva, permitindo o monitoramento contínuo da condição mecânica de equipamentos rotativos \cite{rao2011mechanical}. A vibração em máquinas rotativas resulta da interação complexa entre forças desequilibradas, rigidez estrutural, amortecimento do sistema e excitações externas, manifestando-se como movimentos oscilatórios que podem ser medidos e analisados \cite{girdhar2004practical}.

Em equipamentos industriais, a vibração é caracterizada por sua amplitude, frequência e fase, sendo comumente expressa em unidades de velocidade (mm/s) ou aceleração (m/s²). A análise espectral da vibração permite identificar componentes específicas relacionadas a diferentes fontes de excitação, como desbalanceamento, desalinhamento, folgas mecânicas, defeitos em rolamentos e problemas de lubrificação \cite{mobley2002introduction}.

O desbalanceamento representa uma das principais causas de vibração em equipamentos rotativos, ocorrendo quando o centro de massa não coincide com o eixo de rotação. Este fenômeno gera forças centrífugas proporcionais ao quadrado da velocidade angular, resultando em vibração síncrona com a rotação do equipamento \cite{vance2010rotordynamics}. O desalinhamento entre eixos conectados produz componentes harmônicas características, particularmente na segunda harmônica da frequência de rotação, e pode acelerar significativamente o desgaste de acoplamentos e rolamentos.

A ressonância mecânica constitui outro aspecto crítico na análise de vibração, ocorrendo quando a frequência de excitação coincide com uma frequência natural do sistema. Nestas condições, pequenas forças podem produzir amplitudes de vibração extremamente elevadas, potencialmente causando danos estruturais graves \cite{inman2014engineering}. A identificação e evitação de condições de ressonância são fundamentais para o projeto e operação segura de equipamentos rotativos.

Os rolamentos representam componentes críticos em máquinas rotativas, sendo responsáveis por uma parcela significativa das falhas em equipamentos industriais. Defeitos em rolamentos produzem padrões de vibração característicos, com frequências específicas relacionadas à geometria do rolamento e à velocidade de rotação. A análise envelope e técnicas de demodulação são frequentemente empregadas para detecção precoce de defeitos em rolamentos \cite{randall2011rolling}.

A transmissão de vibração através da estrutura da máquina é influenciada pelas características dinâmicas do sistema, incluindo rigidez, amortecimento e massas envolvidas. Pontos de medição devem ser selecionados estrategicamente para capturar adequadamente os fenômenos de interesse, considerando caminhos de transmissão e frequências de ressonância estrutural \cite{scheffer2004machinery}.

Sistemas de monitoramento contínuo de vibração empregam acelerômetros ou sensores de velocidade instalados permanentemente nos equipamentos, permitindo acompanhamento em tempo real da condição mecânica. Estes sistemas frequentemente incorporam análise automatizada de tendências, detecção de alarmes baseada em limites pré-estabelecidos e capacidades de diagnóstico assistido por algoritmos de reconhecimento de padrões \cite{jardine2006review}.

A norma ISO 10816 estabelece critérios para avaliação de vibração em máquinas rotativas, definindo zonas de operação baseadas em amplitudes de velocidade de vibração. Zona A representa condição nova ou recém-reparada, Zona B indica operação satisfatória, Zona C sugere operação insatisfatória com necessidade de monitoramento intensivo, e Zona D requer ação corretiva imediata \cite{iso10816}.

\section{Moinhos de Rolos Verticais: Princípios e Falhas Características}
\label{sec:moinhos-rolos}

Os moinhos de rolos verticais constituem equipamentos fundamentais na indústria de processamento mineral, sendo amplamente utilizados na produção de cimento, beneficiamento de minérios e processamento de materiais industriais. Estes equipamentos operam através do princípio de moagem por pressão e cisalhamento, onde material alimentado no centro de uma mesa rotativa é submetido à ação de rolos pressurizados hidraulicamente \cite{austin1984introduction}.

A configuração típica de um moinho de rolos vertical compreende uma mesa rotativa horizontal, múltiplos rolos de moagem montados em braços articulados, sistema hidráulico de pressurização, separador de ar interno e sistema de exaustão. A mesa rotativa, acionada por motor elétrico através de redutor, distribui o material alimentado radialmente através da força centrífuga, formando uma camada de material entre a mesa e os rolos \cite{duda1985cement}.

O processo de moagem ocorre pela combinação de forças de compressão e cisalhamento aplicadas pelos rolos sobre o material depositado na mesa. A pressão hidráulica nos rolos, tipicamente variando entre 5-15 MPa, determina a intensidade da moagem e a finura do produto. O material moído é transportado pelo fluxo de ar ascendente até o separador interno, onde partículas finas são coletadas enquanto material grosso retorna à zona de moagem \cite{locher2006cement}.

A estabilidade operacional dos moinhos verticais depende criticamente da manutenção de uma camada uniforme de material na mesa de moagem. Flutuações na alimentação, variações nas características do material ou perturbações no fluxo de ar podem desestabilizar esta camada, resultando em vibrações excessivas e redução da eficiência de moagem \cite{hesse2004process}.

Falhas características em moinhos de rolos incluem desgaste excessivo das superfícies de moagem, problemas no sistema hidráulico de pressurização, desalinhamento de rolos, desgaste de mancais principais e falhas no sistema de vedação. O desgaste das placas de desgaste na mesa e nos rolos representa a principal causa de paradas programadas, exigindo substituição periódica baseada em critérios de desgaste pré-estabelecidos \cite{tromans1989mill}.

O sistema hidráulico de pressurização dos rolos constitui componente crítico para operação estável. Falhas em cilindros hidráulicos, vazamentos internos, problemas no acumulador de nitrogênio ou degradação do fluido hidráulico podem resultar em instabilidade na força de moagem, causando vibrações e redução da produtividade \cite{gearless2010mills}.

Vibrações excessivas em moinhos verticais podem originar-se de múltiplas causas: instabilidade da camada de moagem, desbalanceamento da mesa rotativa, problemas nos mancais principais, interferência entre rolos e mesa, ou excitações externas transmitidas através da estrutura de suporte. A identificação precisa da fonte de vibração requer análise sistemática combinando medições de vibração, análise espectral e correlação com parâmetros operacionais \cite{reichardt2004vertical}.

A manutenção preditiva em moinhos verticais emprega monitoramento contínuo de parâmetros operacionais incluindo vibração, temperatura de mancais, pressões hidráulicas, corrente do motor principal e características do produto moído. A integração destes parâmetros através de sistemas de monitoramento avançados permite detecção precoce de anomalias e otimização das estratégias de manutenção \cite{holderbank1993mill}.

\section{Machine Learning em Manutenção Preditiva}
\label{sec:ml-manutencao}

A manutenção preditiva baseada em machine learning representa uma evolução natural dos sistemas tradicionais de monitoramento de condição, oferecendo capacidades avançadas de reconhecimento de padrões, predição de falhas e otimização de estratégias de manutenção \cite{lee2014prognostics}. A aplicação de técnicas de aprendizado de máquina em dados de monitoramento permite identificar relações complexas entre variáveis operacionais e condição dos equipamentos, frequentemente imperceptíveis através de análise manual tradicional.

Os sistemas de manutenção preditiva tradicionais baseiam-se em análise de limites fixos e tendências de parâmetros individuais, oferecendo capacidade limitada para capturar interações multivariadas complexas. Machine learning supera estas limitações através de algoritmos capazes de processar simultaneamente múltiplas variáveis, identificar padrões não-lineares e adaptar-se automaticamente às características específicas dos equipamentos monitorados \cite{susto2015machine}.

Algoritmos supervisionados de machine learning requerem dados rotulados para treinamento, onde exemplos de condições normais e anômalas são utilizados para construir modelos de classificação ou regressão. Em aplicações de manutenção preditiva, estes rótulos podem representar condições de falha conhecidas, níveis de degradação ou tempo restante até falha. A qualidade e quantidade dos dados rotulados influenciam diretamente a performance dos modelos desenvolvidos \cite{carvalho2019systematic}.

Técnicas de aprendizado não-supervisionado são particularmente valiosas quando dados rotulados são escassos, permitindo detecção de anomalias baseada exclusivamente em padrões de operação normal. Algoritmos como clustering, análise de componentes principais e autoencoders podem identificar condições operacionais atípicas sem necessidade de exemplos prévios de falhas \cite{zhao2019machine}.

O processamento de dados temporais em manutenção preditiva apresenta desafios específicos relacionados à natureza sequencial das observações. Séries temporais de sensores contêm informações sobre tendências de degradação, padrões sazonais e correlações temporais que devem ser adequadamente capturadas pelos algoritmos de machine learning. Técnicas de feature engineering temporal, incluindo estatísticas móveis, transformadas espectrais e análise de autocorrelação, são frequentemente empregadas para extrair características relevantes dos dados temporais \cite{wang2019deep}.

A seleção de características (feature selection) constitui etapa crítica no desenvolvimento de modelos preditivos, especialmente quando lidando com datasets de alta dimensionalidade comuns em aplicações industriais. Técnicas de seleção baseadas em correlação, importância mutual e métodos embedded permitem identificar subconjuntos de variáveis mais relevantes para predição, reduzindo complexidade computacional e melhorando interpretabilidade dos modelos \cite{guyon2003introduction}.

A validação de modelos de machine learning em aplicações de manutenção preditiva deve considerar a natureza temporal dos dados, evitando vazamento de informação futuras (data leakage) durante treinamento. Técnicas de validação cruzada temporal, onde modelos são treinados em períodos passados e testados em períodos futuros, proporcionam estimativas mais realistas da performance em cenários operacionais reais \cite{cerqueira2020evaluating}.

A interpretabilidade dos modelos representa aspecto fundamental em aplicações industriais, onde engenheiros de manutenção necessitam compreender os fatores contribuintes para decisões automatizadas. Técnicas de explicabilidade como importância de características, valores SHAP e análise de sensibilidade permitem insights sobre o comportamento dos modelos e facilitam aceitação por parte dos usuários finais \cite{arrieta2020explainable}.

Sistemas de machine learning para manutenção preditiva devem incorporar capacidades de atualização incremental, permitindo adaptação contínua às condições operacionais em evolução. Algoritmos de online learning e transfer learning facilitam a incorporação de novos dados sem necessidade de retreinamento completo, mantendo relevância dos modelos ao longo do tempo \cite{khamassi2018discussion}.

A integração de modelos de machine learning em sistemas de manutenção existentes requer consideração cuidadosa de aspectos de latência, confiabilidade e manutenibilidade. Arquiteturas de edge computing permitem processamento local de dados de sensores, reduzindo latência e dependência de conectividade de rede, enquanto sistemas na nuvem oferecem capacidades computacionais superiores para modelos complexos \cite{shi2016edge}.

\section{Algoritmos de Machine Learning para Análise de Vibração}
\label{sec:algoritmos-ml}

A seleção apropriada de algoritmos de machine learning para análise de vibração depende das características específicas dos dados, objetivos de predição e requisitos operacionais. Diferentes famílias de algoritmos oferecem vantagens particulares para diferentes tipos de problemas de predição de vibração, desde regressão linear simples até ensemble methods complexos e redes neurais profundas \cite{zhao2019machine}.

Algoritmos de regressão linear representam a base fundamental para muitas aplicações de predição de vibração, oferecendo interpretabilidade superior e baixo custo computacional. A regressão linear assume relação linear entre variáveis preditoras e variável resposta, sendo particularmente efetiva quando esta suposição é válida. Regularização através de Ridge e Lasso regression permite lidar com multicolinearidade e seleção automática de características, respectivamente \cite{hastie2009elements}.

Árvores de decisão e métodos ensemble baseados em árvores, como Random Forest, oferecem capacidade de capturar relações não-lineares complexas entre variáveis de entrada e vibração. Random Forest combina múltiplas árvores de decisão através de bootstrap aggregating (bagging), reduzindo overfitting e fornecendo estimativas de incerteza através da variabilidade entre árvores individuais. A importância de características calculada pelo Random Forest fornece insights valiosos sobre variáveis mais relevantes para predição de vibração \cite{breiman2001random}.

Gradient boosting methods, incluindo XGBoost, LightGBM e CatBoost, representam estado da arte em muitas aplicações de machine learning estruturado. Estes algoritmos constroem modelos ensemble através de combinação sequencial de modelos fracos, onde cada modelo subsequente foca em corrigir erros dos modelos anteriores. XGBoost incorpora regularização avançada e otimizações computacionais que o tornam particularmente efetivo para dados tabulares típicos de aplicações industriais \cite{chen2016xgboost}.

Support Vector Machines (SVM) oferecem capacidades robustas para problemas de regressão através do conceito de margin maximization e uso de kernel functions para mapear dados para espaços de dimensionalidade superior. SVM com kernel RBF pode capturar relações não-lineares complexas, sendo particularmente efetivo em cenários com ruído e outliers. No entanto, o alto custo computacional pode limitar aplicabilidade para datasets grandes \cite{vapnik1998statistical}.

K-Nearest Neighbors (KNN) representa algoritmo baseado em instâncias que realiza predições baseando-se em similaridade com observações de treinamento. Para predição de vibração, KNN pode ser efetivo quando padrões locais são mais importantes que tendências globais. A seleção apropriada do número de vizinhos (k) e métricas de distância é crítica para performance do algoritmo \cite{cover1967nearest}.

Redes neurais artificiais oferecem capacidade universal de aproximação de funções, sendo capazes de modelar relações extremamente complexas entre variáveis de entrada e vibração. Arquiteturas feedforward multicamadas são comumente empregadas para problemas de regressão, enquanto redes recorrentes (LSTM, GRU) são particularmente adequadas para dados sequenciais temporais. O treinamento efetivo de redes neurais requer datasets grandes e consideração cuidadosa de hiperparâmetros para evitar overfitting \cite{goodfellow2016deep}.

Algoritmos robustos como Huber regression são especialmente valiosos em aplicações de vibração onde outliers são comuns devido a condições operacionais atípicas ou ruído de sensores. Huber regression combina propriedades de robustez dos estimadores L1 com eficiência dos estimadores L2, oferecendo performance estável na presença de observações atípicas \cite{huber1964robust}.

Ensemble methods que combinam predições de múltiplos algoritmos base podem oferecer performance superior através de redução de variância e bias. Voting regressors combinam predições através de média simples ou ponderada, enquanto stacking utiliza meta-learners para aprender combinações ótimas. A diversidade entre modelos base é fundamental para efetividade de métodos ensemble \cite{zhou2012ensemble}.

A seleção de hiperparâmetros representa aspecto crítico para otimização de performance de algoritmos de machine learning. Técnicas como grid search, random search e otimização Bayesiana permitem exploração sistemática do espaço de hiperparâmetros. Validação cruzada temporal deve ser empregada para evitar overfitting e fornecer estimativas realistas de performance em dados futuros \cite{bergstra2012random}.

Feature engineering específico para dados de vibração inclui extração de características no domínio da frequência através de FFT, análise de envelope para detecção de defeitos em rolamentos, estatísticas de ordem superior para capturar não-gaussianidade, e características temporais como RMS, kurtosis e crest factor. A qualidade das características extraídas influencia diretamente a performance dos algoritmos de machine learning \cite{lei2020applications}.

\section{Métricas de Avaliação para Modelos Comparativos}
\label{sec:metricas-avaliacao}

A avaliação sistemática de modelos de machine learning para predição de vibração requer utilização de métricas apropriadas que capturem diferentes aspectos de performance, incluindo precisão, robustez, calibração e eficiência computacional. A seleção de métricas deve alinhar-se com objetivos específicos da aplicação e considerar custos relativos de diferentes tipos de erros de predição \cite{botchkarev2018performance}.

O coeficiente de determinação (R²) representa métrica fundamental para avaliação de modelos de regressão, indicando a proporção da variância na variável dependente explicada pelo modelo. R² varia entre 0 e 1, onde valores próximos a 1 indicam alta capacidade explicativa. No entanto, R² pode ser inflado artificialmente pela inclusão de variáveis irrelevantes, sendo necessário considerar versões ajustadas que penalizam complexidade do modelo \cite{steel2013principles}.

Root Mean Square Error (RMSE) quantifica a magnitude típica dos erros de predição, sendo expressa nas mesmas unidades da variável dependente. RMSE penaliza mais fortemente erros grandes devido ao termo quadrático, sendo particularmente sensível a outliers. Esta métrica é valiosa quando erros grandes são especialmente indesejáveis, como em aplicações de segurança crítica \cite{willmott2005advantages}.

Mean Absolute Error (MAE) representa a magnitude média dos erros de predição, oferecendo interpretação mais direta que RMSE. MAE é menos sensível a outliers, fornecendo estimativa mais robusta da performance típica do modelo. A comparação entre RMSE e MAE pode revelar informações sobre a distribuição dos erros, onde RMSE significativamente maior que MAE indica presença de erros grandes ocasionais \cite{willmott2005advantages}.

Mean Absolute Percentage Error (MAPE) normaliza erros pela magnitude da variável dependente, permitindo comparação entre datasets com diferentes escalas. No entanto, MAPE apresenta limitações quando valores verdadeiros são próximos de zero, podendo resultar em valores infinitos ou instáveis. MAPE é particularmente útil para interpretação de resultados por stakeholders não-técnicos \cite{kim2016new}.

A análise de resíduos fornece insights importantes sobre adequação dos modelos através de exame de padrões sistemáticos nos erros de predição. Resíduos homoscedásticos e normalmente distribuídos indicam adequação das suposições do modelo, enquanto padrões sistemáticos podem sugerir necessidade de transformações ou modelos mais complexos. Testes de normalidade como Shapiro-Wilk e análise de heteroscedasticidade são ferramentas valiosas neste contexto \cite{montgomery2012introduction}.

Métricas de robustez avaliam estabilidade de performance dos modelos na presença de ruído, outliers ou mudanças nas condições operacionais. Técnicas como cross-validation com diferentes partições, análise de sensibilidade a perturbações nos dados e avaliação em subconjuntos específicos podem revelar vulnerabilidades dos modelos. A robustez é particularmente crítica em aplicações industriais onde condições operacionais podem variar significativamente \cite{huber2009robust}.

A eficiência computacional engloba tanto tempo de treinamento quanto tempo de inferência dos modelos. Tempo de treinamento é relevante para aplicações que requerem retreinamento frequente, enquanto tempo de inferência é crítico para sistemas de tempo real. Métricas como throughput (predições por segundo) e latência (tempo para uma predição) são particularmente relevantes para deployment em produção \cite{li2020federated}.

Métricas de calibração avaliam quão bem as estimativas de incerteza dos modelos correspondem à frequência real de erros. Modelos bem calibrados produzem intervalos de predição que contêm a proporção esperada de observações reais. Diagramas de calibração e estatísticas como Brier score permitem avaliação quantitativa da calibração \cite{guo2017calibration}.

A análise de overfitting através de comparação entre performance de treinamento e teste é fundamental para avaliação de generalização dos modelos. Diferenças significativas entre estas métricas indicam overfitting, sugerindo necessidade de regularização, redução de complexidade ou aumento da quantidade de dados de treinamento. Learning curves que mostram performance em função do tamanho do dataset podem orientar decisões sobre coleta adicional de dados \cite{domingos2012few}.

Testes estatísticos de significância, como testes t pareados ou testes não-paramétricos de Wilcoxon, permitem avaliação objetiva de diferenças de performance entre modelos. Estes testes consideram variabilidade estatística nas métricas de performance, fornecendo confiança estatística nas comparações. Correções para múltiplas comparações, como correção de Bonferroni, devem ser aplicadas quando comparando múltiplos modelos simultaneamente \cite{demsar2006statistical}.

\section{Trabalhos Relacionados e Estado da Arte}
\label{sec:trabalhos-relacionados}

A aplicação de machine learning para predição de vibração em equipamentos industriais tem sido objeto de crescente interesse na literatura científica, com contribuições significativas abrangendo desde técnicas básicas de regressão até arquiteturas avançadas de deep learning. Esta seção apresenta uma revisão dos principais trabalhos relacionados, destacando metodologias, resultados e limitações identificadas \cite{lei2020applications}.

\citeonline{zhang2018deep} desenvolveram uma arquitetura de rede neural convolucional (CNN) para classificação de falhas em rolamentos baseada em sinais de vibração. O estudo demonstrou superioridade das CNNs em comparação com métodos tradicionais de extração manual de características, alcançando acurácia superior a 99\% em dataset controlado. Limitações incluem avaliação restrita a condições laboratoriais e falta de validação em dados industriais reais com ruído e variabilidade operacional.

\citeonline{chen2019machine} compararam múltiplos algoritmos de machine learning (SVM, Random Forest, Neural Networks) para predição de vibração em turbinas eólicas, utilizando dados de 18 meses de operação. Random Forest apresentou melhor performance (R² = 0.87), superando SVM (R² = 0.82) e redes neurais (R² = 0.84). O estudo destacou a importância de feature engineering específico para o domínio, incluindo características espectrais e estatísticas de ordem superior.

\citeonline{wang2020predictive} propuseram framework integrado combinando análise de componentes principais (PCA) para redução de dimensionalidade com ensemble de algoritmos (Random Forest, XGBoost, Support Vector Regression) para predição de vibração em compressores centrífugos. A abordagem ensemble alcançou RMSE 15\% inferior aos modelos individuais, demonstrando benefícios da combinação de algoritmos complementares.

\citeonline{liu2019vibration} investigaram aplicação de Long Short-Term Memory (LSTM) networks para predição de vibração em mancais de máquinas rotativas, considerando dependências temporais nos dados. A rede LSTM superou modelos tradicionais de regressão (R² = 0.94 vs 0.78), especialmente em horizontes de predição de médio prazo (30-60 minutos). Limitações incluem alto custo computacional e necessidade de grandes quantidades de dados sequenciais.

\citeonline{zhao2019intelligent} desenvolveram sistema híbrido combinando Wavelet Transform para pré-processamento de sinais com ensemble de algoritmos (Gradient Boosting, Random Forest, Neural Networks) para diagnóstico de falhas em moinhos de rolos. O sistema alcançou acurácia de 95\% na classificação de condições operacionais, superando abordagens tradicionais baseadas em análise espectral manual.

\citeonline{kumar2020machine} realizaram estudo comparativo abrangente de algoritmos de machine learning para manutenção preditiva em equipamentos rotativos, avaliando 12 algoritmos diferentes em datasets de 5 tipos de equipamentos. XGBoost e Random Forest emergiram como algoritmos mais consistentes, apresentando boa performance em diferentes tipos de equipamentos e condições operacionais.

\citeonline{li2018fault} propuseram arquitetura de deep autoencoder para detecção de anomalias em vibração de equipamentos rotativos, focando em detecção não-supervisionada de condições anômalas. A abordagem demonstrou capacidade de detectar 92\% das falhas incipientes com taxa de falsos positivos inferior a 5\%, superando métodos tradicionais baseados em limites fixos.

\citeonline{ahmed2021comparative} conduziram análise comparativa de técnicas de feature selection para predição de vibração, avaliando métodos baseados em correlação, informação mutual e importância por árvores. Métodos embedded (baseados em Random Forest) apresentaram melhor trade-off entre redução de dimensionalidade e preservação de performance preditiva.

Lacunas identificadas na literatura incluem: (1) limitada validação em dados industriais reais com condições operacionais variáveis; (2) falta de comparações sistemáticas considerando múltiplas métricas de performance e eficiência computacional; (3) inadequada consideração de aspectos de interpretabilidade e explicabilidade dos modelos; (4) insuficiente análise de robustez a ruído e outliers comuns em ambientes industriais \cite{carvalho2019systematic}.

O presente trabalho contribui para o estado da arte através de: (1) comparação sistemática de múltiplos algoritmos em dados industriais reais de moinhos de rolos verticais; (2) avaliação abrangente considerando múltiplas métricas de performance, robustez e eficiência; (3) análise detalhada de feature importance e interpretabilidade dos modelos; (4) validação temporal rigorosa evitando data leakage; (5) consideração específica de características operacionais de moinhos verticais na indústria cimenteira \cite{susto2015machine}.