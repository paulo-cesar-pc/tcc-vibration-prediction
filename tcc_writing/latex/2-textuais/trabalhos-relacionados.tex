\chapter{Trabalhos Relacionados}
\label{chap:trabalhos-relacionados}

Neste capítulo serão apresentados trabalhos que estão conectados a essa pesquisa.

\section{Vibration Energy Coupling Behavior of Rolling Mills under Double Disturbance Conditions \cite{vibration-energy-coupling-behavior}}
\label{sec:trabalho-relacionado-a}

O artigo explora as variáveis que influenciam o processo de vibração em laminadores, destacando a complexidade do comportamento dinâmico sob condições de distúrbio. Os autores identificam variáveis críticas que afetam a energia de vibração e a estabilidade do sistema.

Uma das principais variáveis discutidas é a flutuação da força de laminação. O estudo aponta que as flutuações nas forças aplicadas durante o processo de laminação têm um impacto significativo no fluxo de energia de vibração. Aumentos nas flutuações de força resultam em um aumento correspondente na energia de vibração, indicando que a estabilidade do laminador pode ser comprometida por variações inesperadas nas forças de laminação.

Outra variável importante é o torque de pré-carga. Embora as mudanças no torque de pré-carga não afetem diretamente a amplitude do fluxo de energia de vibração, a interação entre o torque e as flutuações de força pode influenciar a dinâmica do sistema. O estudo sugere que a otimização do torque de pré-carga pode ajudar a controlar as vibrações.

Além disso, o damping (ou amortecimento) do sistema é uma variável crucial. O artigo mostra que, com um coeficiente de amortecimento entre 0,001 e 0,01, a energia de vibração diminui significativamente sob certas frequências de excitação. Quando o coeficiente de amortecimento é aumentado para entre 0,01 e 0,1, a redução da energia de vibração se torna ainda mais pronunciada, indicando que um bom controle do amortecimento pode ser uma estratégia eficaz para mitigar vibrações.

A largura da tira e o módulo da tira também são destacados como variáveis que afetam o comportamento de vibração. O aumento do módulo da tira está associado a um aumento no fluxo de energia de vibração no sistema de acionamento principal, enquanto a variação na largura da tira tem um impacto mais significativo no sistema vertical.

Por fim, o ângulo de fase entre o torque de laminação e as flutuações de torque é identificado como uma variável que apresenta um padrão de "V" no fluxo de energia de vibração, com um mínimo de energia em ângulos de fase específicos. Essa relação complexa entre as variáveis destaca a necessidade de um controle cuidadoso para otimizar o desempenho do laminador e reduzir as vibrações indesejadas.

\section{Multi-scale reconstruction of rolling mill vibration signal based on fuzzy entropy clustering \cite{multi-scale-reconstruction}}
\label{sec:trabalho-relacionado-b}

O artigo aborda a complexidade das vibrações em moinhos de laminação, enfatizando como diversas variáveis afetam esses sinais. Durante a operação dos moinhos, as vibrações são influenciadas por fatores como condições de trabalho, variações na carga, e a interação entre os componentes mecânicos do equipamento. Essas influências resultam em sinais de vibração que apresentam características não estacionárias, tornando a análise e a previsão desafiadoras.

Os autores destacam que as vibrações verticais são particularmente pronunciadas durante o processo de laminação a frio, o que justifica a coleta de dados focada nesse sentido. O uso do sensor de vibração AC104-1A permite a captura precisa dessas vibrações, mas os sinais coletados ainda são afetados por ruídos e interferências, que podem prejudicar a qualidade das informações obtidas.

Para lidar com esses desafios, o estudo propõe um método de reconstrução de sinais em múltiplas escalas, que combina a entropia fuzzy com o algoritmo de clustering Gath-Geva. Este método visa decompor os sinais de vibração em componentes intrínsecos (IMFs), permitindo uma análise mais detalhada das características de cada componente. A decomposição modal ajuda a isolar as influências específicas que afetam as vibrações, como flutuações na tensão do material, variações na velocidade de operação e desbalanceamentos mecânicos.

Os resultados mostram que a abordagem proposta não apenas melhora a qualidade dos sinais reconstruídos, mas também facilita a identificação das variáveis que impactam as vibrações. A pesquisa conclui que a análise das vibrações em moinhos de laminação, considerando suas características não estacionárias e as variáveis que as afetam, é crucial para o desenvolvimento de modelos preditivos mais precisos, contribuindo para a manutenção eficiente e a operação segura dos equipamentos. Essa metodologia pode ser aplicada para otimizar processos e reduzir falhas, melhorando a confiabilidade dos moinhos.